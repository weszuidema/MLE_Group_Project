\documentclass[11pt,]{article}
\usepackage[]{mathpazo}
\usepackage{amssymb,amsmath}
\usepackage{ifxetex,ifluatex}
\usepackage{fixltx2e} % provides \textsubscript
\ifnum 0\ifxetex 1\fi\ifluatex 1\fi=0 % if pdftex
  \usepackage[T1]{fontenc}
  \usepackage[utf8]{inputenc}
\else % if luatex or xelatex
  \ifxetex
    \usepackage{mathspec}
  \else
    \usepackage{fontspec}
  \fi
  \defaultfontfeatures{Ligatures=TeX,Scale=MatchLowercase}
  \newcommand{\euro}{€}
\fi
% use upquote if available, for straight quotes in verbatim environments
\IfFileExists{upquote.sty}{\usepackage{upquote}}{}
% use microtype if available
\IfFileExists{microtype.sty}{%
\usepackage{microtype}
\UseMicrotypeSet[protrusion]{basicmath} % disable protrusion for tt fonts
}{}
\usepackage[margin=1in]{geometry}
\usepackage{hyperref}
\PassOptionsToPackage{usenames,dvipsnames}{color} % color is loaded by hyperref
\hypersetup{unicode=true,
            pdftitle={MLE Proposal},
            pdfauthor={Ellen Ahlness and Wesley Zuidema},
            pdfborder={0 0 0},
            breaklinks=true}
\urlstyle{same}  % don't use monospace font for urls
\usepackage{natbib}
\bibliographystyle{plainnat}
\usepackage{graphicx,grffile}
\makeatletter
\def\maxwidth{\ifdim\Gin@nat@width>\linewidth\linewidth\else\Gin@nat@width\fi}
\def\maxheight{\ifdim\Gin@nat@height>\textheight\textheight\else\Gin@nat@height\fi}
\makeatother
% Scale images if necessary, so that they will not overflow the page
% margins by default, and it is still possible to overwrite the defaults
% using explicit options in \includegraphics[width, height, ...]{}
\setkeys{Gin}{width=\maxwidth,height=\maxheight,keepaspectratio}
\setlength{\parindent}{0pt}
\setlength{\parskip}{6pt plus 2pt minus 1pt}
\setlength{\emergencystretch}{3em}  % prevent overfull lines
\providecommand{\tightlist}{%
  \setlength{\itemsep}{0pt}\setlength{\parskip}{0pt}}
\setcounter{secnumdepth}{0}

%%% Use protect on footnotes to avoid problems with footnotes in titles
\let\rmarkdownfootnote\footnote%
\def\footnote{\protect\rmarkdownfootnote}

%%% Change title format to be more compact
\usepackage{titling}

% Create subtitle command for use in maketitle
\newcommand{\subtitle}[1]{
  \posttitle{
    \begin{center}\large#1\end{center}
    }
}

\setlength{\droptitle}{-2em}
  \title{MLE Proposal}
  \pretitle{\vspace{\droptitle}\centering\huge}
  \posttitle{\par}
  \author{Ellen Ahlness and Wesley Zuidema}
  \preauthor{\centering\large\emph}
  \postauthor{\par}
  \predate{\centering\large\emph}
  \postdate{\par}
  \date{October 13, 2016}


\usepackage{setspace}
\setlength\parindent{24pt}

% Redefines (sub)paragraphs to behave more like sections
\ifx\paragraph\undefined\else
\let\oldparagraph\paragraph
\renewcommand{\paragraph}[1]{\oldparagraph{#1}\mbox{}}
\fi
\ifx\subparagraph\undefined\else
\let\oldsubparagraph\subparagraph
\renewcommand{\subparagraph}[1]{\oldsubparagraph{#1}\mbox{}}
\fi

\begin{document}
\maketitle

NOTE: We apologize for the random placement of our figures. We are still
mastering R Markdown and lacked the experience to get everything on the
correct pages.

Our project will explore trends in the relationship between individual
values and political party preferences over time. In particular, we are
interested in the question of whether individuals whose primary
political concern is immigration will be more drawn to far right parties
than individuals who are primarily concerned with other political
issues\footnote{Berning, Carl C., and Elmar Schlueter. 2016. ``The
  Dynamics of Radical Right-Wing Populist Party Preferences and
  Perceived Group Threat: A Comparative Panel Analysis of Three
  Competing Hypotheses in the Netherlands and Germany.'' Social Science
  Research 55: 83--93.}. The data come from the World Values
Survey\footnote{World Values Survey.
  \url{http://www.worldvaluessurvey.org/WVSDocumentationWVL.jsp}}. Our
project focuses on using professed individual values to explain an
affinity for far-right parties in Europe and the United States.
Specifically, we are interested in how an individual describes
themselves (E047-E056), their favored immigration policy (E143), what
they believe the most serious problems facing the world are
(E238-E239\_ES), and what they believe the most serious problems facing
their state are (E240-E241\_ES). We hypothesize, following in the
footsteps of other research, that an outsized concern with immigration
will make an individual more likely to prefer far-right parties.

All our data is measured at the individual level. Our independent
(treatment) variables, measured by questions about a person's primary
policy concerns, are categorical; there is no rank among what
individuals believe is the biggest threat against the world and their
state. The dependent variable, party identification, is ordinal. While
there is no inherent rank in political party preference, they can be
organized on a scale from `far left' to `far right,' where far right
parties are ultra-nationalist in character that prefer, among other
things, a radical reduction in the flow of immigration and a nationalist
challenge to secular multiculturalism. Other covariates of interest that
we will use as controls are gender, income, age, and marital status, all
of which have been shown to play a role in party identification and
preference.

Some of our covariates of interest appear to be normally distributed. In
particular, immigration policy preferences seem to be distributed
normally around the mean. We think it is likely that the variables that
use ranked, discrete scales (e.g., 1 to 10) will be distributed normally
due to our large sample size and the central limit theorem.
Nevertheless, it is possible that a different distribution might
describe the data more accurately. For instance, acceptance of change
seems to have the majority of responses fall at either extreme or at the
exact median. We are not sure at this point what distribution best
approximates this data. As of right now, we have not cleaned the data
enough to provide a histogram of party preferences, but we expect that
any parties ranked from left to right will be distributed normally
around the centrist parties. If a society is particularly polarized,
that assumption may prove wrong.

\begin{figure}[htbp]
\centering
\includegraphics{02_Proposal_files/figure-latex/unnamed-chunk-3-1.pdf}
\caption{Immigration Policy}
\end{figure}

\begin{figure}[htbp]
\centering
\includegraphics{02_Proposal_files/figure-latex/unnamed-chunk-4-1.pdf}
\caption{Acceptance of Change}
\end{figure}

With our categorical and ordinal data, two methods stand out for
analysis. The first is logistic regression, and the second is
non-parametric correlation. Since our outcome variable is categorical,
we will need to use a model that is appropriate for discrete outcome
variables. At this point, we will need to get a better grasp of the data
before we are sure which model will be appropriate.

\pagebreak

The covariates of interest we will use as controls follow a number of
their own unique distributions. Age seems to follow a Poisson
distribution. Income level seems to be distributed fairly normally with
a long right tail, respresenting outliers who make exceptionally large
amounts of money. Sex seems to follow the Bernoulli distribution,
although our data do not include individuals who intersex or who do not
identify with the gender assigned at birth. While most individuals are
either single or never married, there are a significant number of
individuals who are divorced, widowed, or living with a party as if they
were married. Our analysis will need to treat this as a categorical
variable without rank.

\begin{figure}[htbp]
\centering
\includegraphics{02_Proposal_files/figure-latex/unnamed-chunk-5-1.pdf}
\caption{Age}
\end{figure}

\begin{figure}[htbp]
\centering
\includegraphics{02_Proposal_files/figure-latex/unnamed-chunk-6-1.pdf}
\caption{Income Level}
\end{figure}

\begin{figure}[htbp]
\centering
\includegraphics{02_Proposal_files/figure-latex/unnamed-chunk-7-1.pdf}
\caption{Sex}
\end{figure}

\begin{figure}[htbp]
\centering
\includegraphics{02_Proposal_files/figure-latex/unnamed-chunk-8-1.pdf}
\caption{Marital Status}
\end{figure}

\end{document}
